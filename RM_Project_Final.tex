\documentclass[10pt,]{article}
\usepackage{lmodern}
\usepackage{amssymb,amsmath}
\usepackage{ifxetex,ifluatex}
\usepackage{fixltx2e} % provides \textsubscript
\ifnum 0\ifxetex 1\fi\ifluatex 1\fi=0 % if pdftex
  \usepackage[T1]{fontenc}
  \usepackage[utf8]{inputenc}
\else % if luatex or xelatex
  \ifxetex
    \usepackage{mathspec}
    \usepackage{xltxtra,xunicode}
  \else
    \usepackage{fontspec}
  \fi
  \defaultfontfeatures{Mapping=tex-text,Scale=MatchLowercase}
  \newcommand{\euro}{€}
\fi
% use upquote if available, for straight quotes in verbatim environments
\IfFileExists{upquote.sty}{\usepackage{upquote}}{}
% use microtype if available
\IfFileExists{microtype.sty}{%
\usepackage{microtype}
\UseMicrotypeSet[protrusion]{basicmath} % disable protrusion for tt fonts
}{}
\usepackage[margin=1cm]{geometry}
\usepackage{graphicx}
\makeatletter
\def\maxwidth{\ifdim\Gin@nat@width>\linewidth\linewidth\else\Gin@nat@width\fi}
\def\maxheight{\ifdim\Gin@nat@height>\textheight\textheight\else\Gin@nat@height\fi}
\makeatother
% Scale images if necessary, so that they will not overflow the page
% margins by default, and it is still possible to overwrite the defaults
% using explicit options in \includegraphics[width, height, ...]{}
\setkeys{Gin}{width=\maxwidth,height=\maxheight,keepaspectratio}
\ifxetex
  \usepackage[setpagesize=false, % page size defined by xetex
              unicode=false, % unicode breaks when used with xetex
              xetex]{hyperref}
\else
  \usepackage[unicode=true]{hyperref}
\fi
\hypersetup{breaklinks=true,
            bookmarks=true,
            pdfauthor={Rabindra Swamidasan},
            pdftitle={Regression Modeling Project},
            colorlinks=true,
            citecolor=blue,
            urlcolor=blue,
            linkcolor=magenta,
            pdfborder={0 0 0}}
\urlstyle{same}  % don't use monospace font for urls
\setlength{\parindent}{0pt}
\setlength{\parskip}{6pt plus 2pt minus 1pt}
\setlength{\emergencystretch}{3em}  % prevent overfull lines
\setcounter{secnumdepth}{0}

%%% Use protect on footnotes to avoid problems with footnotes in titles
\let\rmarkdownfootnote\footnote%
\def\footnote{\protect\rmarkdownfootnote}

%%% Change title format to be more compact
\usepackage{titling}
\setlength{\droptitle}{-2em}
  \title{Regression Modeling Project}
  \pretitle{\vspace{\droptitle}\centering\huge}
  \posttitle{\par}
  \author{Rabindra Swamidasan}
  \preauthor{\centering\large\emph}
  \postauthor{\par}
  \predate{\centering\large\emph}
  \postdate{\par}
  \date{Sunday, May 24, 2015}




\begin{document}

\maketitle


\subsection{Motor Trend Data Mileage
Analysis}\label{motor-trend-data-mileage-analysis}

\subsubsection{1. Executive Summary}\label{executive-summary}

We analyse data extracted from the 1974 Motor Trend US magazine
consisting of fuel consumption(mpg) and 10 other parameters for 32 cars.
The analysis focuses on the question whether automatic or manual
transmissions are better for mpg and to what quantifiable extent. To
this end, we develop a linear regression model to explain the variation
in the mileage of the cars in the data set based on the other 10
parameters.

The short answer turns out to be: It depends \ldots{} on the weight of
the car. All other factors being the same, the expected mileage (in mpg)
of a car with manual transmission exceeds that of a car with automatic
transmission by 14.08 - 4.14 x the weight of the car in 1,000 lbs. In
other words, below approximately 3,400 lbs, cars with manual
transmission are more fuel efficient than automatics, all other factors
being the same. Above 3,400 lbs, automatics are less thirsty.

\subsubsection{2. Exploratory Data
Analysis}\label{exploratory-data-analysis}

A brief look at the data set shows the variable names and some values.
Transmission is denoted by \textbf{am}, where 0 = automatic, 1 = manual.

\begin{verbatim}
##                mpg cyl disp  hp drat    wt  qsec vs am gear carb
## Mazda RX4     21.0   6  160 110 3.90 2.620 16.46  0  1    4    4
## Mazda RX4 Wag 21.0   6  160 110 3.90 2.875 17.02  0  1    4    4
## Datsun 710    22.8   4  108  93 3.85 2.320 18.61  1  1    4    1
\end{verbatim}

To narrow our exploration of the data in line with our goal, we recall
from elementary physics that the kinetic energy of a body is
proportional to its mass. Therefore, the variable \textbf{wt} must be an
important element affecting \textbf{mpg}. To examine this relationship,
we plot \textbf{mpg} vs \textbf{wt}, color coded for \textbf{am}, our
primary variable of interest, with separate fitted lines for Automatic
and Manual transmissions.

Fig 1 in the Appendix shows this plot. We see from Fig 1 that the lines
for Automatic and Manual transmissions have different slopes, and
intersect near the center of the plot. This indicates a dependence of
\textbf{mpg} on an interaction between \textbf{wt} and \textbf{am}. If
there were no interaction, these lines would be parallel. Because they
intersect near the center of the plot, there is a good chance that
manuals may have better mileage in some weight range, but lower mileage
than automatics in some other range.

\subsubsection{3. Model Selection
Strategy}\label{model-selection-strategy}

Our model selection follows the iterative methodology below.

\begin{enumerate}
\def\labelenumi{\arabic{enumi}.}
\itemsep1pt\parskip0pt\parsep0pt
\item
  Start with the model we would like to test. This includes:

  \begin{itemize}
  \itemsep1pt\parskip0pt\parsep0pt
  \item
    A regressor, \textbf{wt}, that we know we should include -- a
    \textbf{\emph{known known}}.
  \item
    A regressor, \textbf{am}, that we would like to include -- a
    \textbf{\emph{known unknown}}.
  \item
    The interaction term, \textbf{wt} * \textbf{am}, on the basis of the
    evidence in Fig 1.
  \end{itemize}
\item
  Verify from the fit that the p-values for all regressor terms are
  below 1\%. If not, discard the offending regressor.
\item
  Build the next model:

  \begin{itemize}
  \itemsep1pt\parskip0pt\parsep0pt
  \item
    Find the correlation of the remaining covariates to the residual
    variance of the current model.
  \item
    Choose the covariate with the highest correlation if the
    \textbar{}correlation\textbar{} \textgreater{}= 0.50.
  \item
    If the covariates all have \textbar{}correlation\textbar{}
    \textless{} 0.50, we do not attempt further improvement. Go to Step
    5.
  \end{itemize}
\item
  Fit the linear model with the new regressor, go to step 2 to check
  p-values.
\item
  Perform an ANOVA test to compare the nested models.
\end{enumerate}

\subsubsection{4. Model Development}\label{model-development}

Execution of the strategy detailed above results in a rapid convergence.

\begin{enumerate}
\def\labelenumi{\arabic{enumi}.}
\itemsep1pt\parskip0pt\parsep0pt
\item
  \textbf{Model 1}: Regressors -- \textbf{wt} + \textbf{am} +
  \textbf{wt} * \textbf{am}; p-values for all regressors were less than
  1\%; Multiple R-squared = 0.833, Adjusted R-squared = 0.8151.
  \textbf{qsec} is the only covariate with \textbar{}r\textbar{}
  \textgreater{} 0.50, with r = 0.51. So, we include it in Model 2.
\item
  \textbf{Model 2}: Regressors -- \textbf{wt} + \textbf{am} +
  \textbf{wt} * \textbf{am} + \textbf{qsec}; p-values for all regressors
  were less than 1\%; Multiple R-squared = 0.8959, Adjusted R-squared =
  0.8804. The covariate with the highest correlation is \textbf{gear}
  with r = 0.09. So, we stop.
\end{enumerate}

The summary results of Model 2 and the ANOVA results are below. A plot
of the results of Model 2 are in Fig 2 in the Appendix.

\begin{verbatim}
## 
## Call:
## lm(formula = mpg ~ wt + am + qsec + wt:am)
## 
## Residuals:
##     Min      1Q  Median      3Q     Max 
## -3.5076 -1.3801 -0.5588  1.0630  4.3684 
## 
## Coefficients:
##             Estimate Std. Error t value Pr(>|t|)    
## (Intercept)    9.723      5.899   1.648 0.110893    
## wt            -2.937      0.666  -4.409 0.000149 ***
## am            14.079      3.435   4.099 0.000341 ***
## qsec           1.017      0.252   4.035 0.000403 ***
## wt:am         -4.141      1.197  -3.460 0.001809 ** 
## ---
## Signif. codes:  0 '***' 0.001 '**' 0.01 '*' 0.05 '.' 0.1 ' ' 1
## 
## Residual standard error: 2.084 on 27 degrees of freedom
## Multiple R-squared:  0.8959, Adjusted R-squared:  0.8804 
## F-statistic: 58.06 on 4 and 27 DF,  p-value: 7.168e-13
\end{verbatim}

\begin{verbatim}
## Analysis of Variance Table
## 
## Model 1: mpg ~ wt * am
## Model 2: mpg ~ wt + am + qsec + wt:am
##   Res.Df    RSS Df Sum of Sq      F   Pr(>F)    
## 1     28 188.01                                 
## 2     27 117.28  1    70.731 16.284 0.000403 ***
## ---
## Signif. codes:  0 '***' 0.001 '**' 0.01 '*' 0.05 '.' 0.1 ' ' 1
\end{verbatim}

The low p-values of the regressors (all \textless{}\textless{} 1\%) and
their t-statistics indicate that there is a low probabilty that any of
them are superfluous. The low ANOVA p-value indicates that the regressor
added in Model 2 contributes to the explanatory power of the model. Fig
2 shows that the the residuals are randomly and homoskedastically
distributed about the fitted line. The Q-Q plot shows that the residuals
have a close to normal distribution. The Leverage plot indicates that
all residuals are within a Cook's distance of 0.5.

\subsubsection{5. Conclusions}\label{conclusions}

Using the coefficients from the results above, the final model is
formalized in the equation below, which explains 89.59\% of the
variation in \textbf{mpg}. An error term is included to account for the
residuals.

\textbf{mpg} = 9.72 - 2.94 * \textbf{wt} + 14.08 * \textbf{am} - 4.14 *
\textbf{wt} * \textbf{am} + 1.02 * \textbf{qsec} + \(\epsilon\)

By setting \textbf{am} = 1 (for manual) and \textbf{am} = 0 (for
automatic) in this equation and taking the difference, we find that
\textbf{mpg} for manual cars is greater than \textbf{mpg} for automatics
by 14.08 - 4.14 * \textbf{wt}. This difference is 0 at the crossover
point: \textbf{wt} = (14.08 / 4.14) x 1000 lbs = 3399.7 lbs. Above this
weight automatics are more fuel efficient than manual transmissions.

\newpage

\subsubsection{6. Appendix}\label{appendix}

\includegraphics{RM_Project_Final_files/figure-latex/unnamed-chunk-4-1.pdf}

\textbf{Fig 1} shows the clear interaction between weight (\textbf{wt})
and transmission type (\textbf{am}) in affecting mileage. If this
interaction was not present, the red (manual) and black (automatic)
fitted lines would have been parallel. For example, at a weight of 2000
lbs, manual has better mileage than automatic. But, at a weight of 3500
lbs, automatic has better mileage. Without interaction, the difference
between manual and automatic would have been the same at 2000 lbs and
3500 lbs (and elsewhere), i.e.~the lines would be parallel.

\includegraphics{RM_Project_Final_files/figure-latex/unnamed-chunk-5-1.pdf}

\section{Fig 2: Residual plot of Model
2}\label{fig-2-residual-plot-of-model-2}

\textbf{NOTE}: This report was authored in R Markdown and compiled to
pdf using pdflatex (via knitr). To view the raw source, please visit the
\href{https://github.com/rswamidasan/Regression-Models}{GitHub repo}
associated with this project.

\end{document}
